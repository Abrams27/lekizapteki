%& --translate-file latin2pl
\documentclass{article}

\usepackage{polski}
\usepackage{hyperref}
\usepackage[T1]{fontenc}
\usepackage[utf8]{inputenc}
\usepackage{listings}
\hypersetup{
colorlinks=true,
urlcolor=blue,
}
\setcounter{section}{-1}
\lstset{
basicstyle=\fontsize{7}{7}\selectfont\ttfamily,
}
\usepackage{graphicx}
\graphicspath{ {./images/} }

\title{
Lekizapteki\\
\large architektura}
\author{Marcin Abramowicz \and Mateusz Danowski \and Dawid Jamka \and Tomasz Patyna}


\begin{document}
  \maketitle

  \section{Dziennik zmian}
    \begin{tabular}{|c|c|c|}
      Nr iteracji & Data & Opis zmian \\
      \hline
      1. & 25.03.2020 & Utworzenie dokumentu oraz jego pierwsza wersja. \\
    \end{tabular}

  \section{Ogólna struktura}
    Aplikacja składa się z frontendu WWW napisanego w
    \href{https://angular.io}{Angularze}, backendu napisanego w
    \href{https://spring.io}{Spring Framewok} oraz bazy danych
    \href {https://www.h2database.com/html/main.html}{H2}.

  \section{Komunikacja}
    \subsection{Backend - Baza Danych}
    Backend komunikuje się z lokalną bazą danych dzięki
    \href{https://hibernate.org}{Hibernate} oraz
    \href{https://spring.io/projects/spring-data-jpa} {Spring Jpa}.
    Framework zajmuje się konstruowaniem zapytań, oraz tworzeniem schematu bazy danych.

    \subsection{Frondend - Backend}
    Komunikacja odbywa się protokołem HTTP, serwer dostarcza
    \href{https://en.wikipedia.org/wiki/Representational_state_transfer}{RESTful API},
    obsługując rządania w formacie
    \href{https://en.wikipedia.org/wiki/JSON}{json}.
    Serwer jest dostepny na
    \href{http://students.mimuw.edu.pl:7312}{serwerze Students na porcie `7312`},
    natomiast frontend wykonuje zapytania za pomocą
    \href{https://angular.io/guide/http}{httpClient'a}.
    Nie ma stosowanych żadnych form zabezpieczeń, rządania i odpowiedzi nie sa szyfrowane, ani nie ma wymiany tokenów.
    API backendu jest publiczne, więc każdy dysponujący linkiem może z niego korzystać.
    Backend akceptuje jedynie zapytania w formacie \texttt{application/json},
    a odpowiedzi również są jedynie w formacie \texttt{application/json}
    Oznacza to, że zapytania do backendu powinny być z nagłowkiem: \texttt{Accept: application/json},
    natomiast odpowiedzi, będą z nagłówkiem: \texttt{Content-Type: application/json}.

    \subsubsection{Endpointy udostepniane przez backend}
    \noindent
    \begin{minipage}{.45\textwidth}
     \begin{lstlisting}
       - GET /lekizapteki/diseases

         Response:
         HTTP/1.1 200 (OK)
         Body:
         [{
           "id": Long,
           "name":"String"
         }]


       - GET /lekizapteki/medicines

         Parameters:
         diseaseId: Long (optional)

         Responses:
         HTTP/1.1 200 (OK)
         Body:
         [{
           "ean": "String",
           "name": "String",
           "dose": "String"
         }]

         HTTP/1.1 404 (Not Found)
         Message: "Nieprawidlowy numer EAN"

         HTTP/1.1 404 (Not Found)
         Message: "Nieprawidlowa jednostka chorobowa"
     \end{lstlisting}
    \end{minipage}\hfill
    \begin{minipage}{.45\textwidth}
      \begin{lstlisting}
        - GET /lekizapteki/medicines/identical

          Parameters:
          ean: String
          diseaseId: Long (optional)

          Responses:
          HTTP/1.1 200 (OK)
          Body:
          [{
            "ean": "String",
            "name": "String",
            "dose": "String"
          }]

          HTTP/1.1 404 (Not Found)
          Message: "Nieprawidlowy numer EAN"

          HTTP/1.1 404 (Not Found)
          Message: "Lek o podanym numerze EAN nie jest
            przepisywany na wybrana jednostke chorobowa"
      \end{lstlisting}
    \end{minipage}

  \section{Frontend}
  Frontend jest dostepny
  \href{http://students.mimuw.edu.pl:1237}{serwerze Students na porcie `1237`} i jest publiczny,
  każdy dysponujący linkiem może z niego korzystać.
  Użytkowanie nie wymaga zakładania konta, ani autoryzacji.

  Frontend posiada web serwis oraz modele odpowiadające odbieranym danyn z backendu.
  Do tworzenia elementów używane są wbudowane komponenty oraz odpowiednie komponenty z
  \href{https://ng-bootstrap.github.io/#/home}{Bootstrap Widgets}.

    \subsection{Makiety}
      \subsubsection{Wybór choroby}
      \includegraphics[width=8cm, height=5cm]{lekizapteki-wybor-choroby}

      \subsubsection{Wybór leku za pomocą nazwy i numeru EAN}
      \includegraphics[width=6.4cm, height=4cm]{lekizapteki-wybor-leku-nazwa}
      \includegraphics[width=6.4cm, height=4cm]{lekizapteki-wybor-leku-ean}

      \subsubsection{Wyniki i rozwinięte szczegóły dla pozycji}
      \includegraphics[width=6.4cm, height=4cm]{lekizapteki-leki-identyczne}
      \includegraphics[width=6.4cm, height=4cm]{lekizapteki-leki-identyczne-rozwiniete}

  \section{Backend}
  Backend jest podzielony na dwa moduły: ``API'' i ``APP''.

    \subsection{Modul API}
    Zawiera wystawione endpoint'y odpowiadające tym przedstawionym w punkcie 2.1.1.
    Są one w postaci interface'ów, klas z modelami oraz wyjatków rzucanych przy nieodpowiednich danych.

    \subsection{Modul APP}
    Zawiera implementacje interface'ów z modulu API.
    Każdy interface jest implementowany przez kontroler.
    Kontrolery maja wstrzykniete jedynie klasy służące do wykonywania pojedyńczy endpoint (Command Pattern).

    Encje i relacje bazy danych sa odwzorowane klasami z odpowiednimi adnotacjami.
    Zapytania do bazy danych odbywają się przez interface, implementowany przez framework,
    w którym nazwy metod są mapowane na zapytania do bazy danych.

    Dane z rozporządzeń są pobierane TODO, w serwisie który jest uruchamiany zaraz po powstaniu kontekstu aplikacji.
    Może on pracować w dwóch trybach, które są konfigurowane w pliku \texttt{application.properties}.
    Pierwszym trybem jest pobieranie plików z podanego w konfiguracji linku,
    drugim jest pobieranie z lokalnego źródła, do którego ścieżka jest podana w konfiguracji.
    Parsuje on pliki z arkuszami Excel, waliduje, filtruje spełniające założenia serwisu oraz mapuje na encje w bazie danych.
  \section{Baza Danych}
  \noindent
  \begin{minipage}{.45\textwidth}
    \begin{lstlisting}
      CREATE TABLE DISEASE (
      id IDENTITY NOT NULL,
      name TEXT NOT NULL,

      CONSTRAINT disease_pk PRIMARY KEY (id)
      );

      CREATE TABLE INGREDIENT (
      id IDENTITY NOT NULL,
      name TEXT NOT NULL,

      CONSTRAINT ingredient_pky PRIMARY KEY (id)
      );

      CREATE TABLE DOSE (
      id IDENTITY NOT NULL,
      dose TEXT NOT NULL,

      CONSTRAINT dose_pk PRIMARY KEY (id)
      );

      CREATE TABLE FORM (
      id IDENTITY NOT NULL,
      name TEXT NOT NULL,

      CONSTRAINT form_pk PRIMARY KEY (id)
      );

      CREATE TABLE PACKAGE (
      id IDENTITY NOT NULL,
      content TEXT NOT NULL,

      CONSTRAINT package_pk PRIMARY KEY (id)
      );
    \end{lstlisting}
  \end{minipage}\hfill
  \begin{minipage}{.45\textwidth}
    \begin{lstlisting}
      CREATE TABLE PRICING (
      id IDENTITY NOT NULL,
      sale_price NUMERIC NOT NULL,
      retail_price NUMERIC NOT NULL,
      total_funding NUMERIC NOT NULL,
      charge_factor NUMERIC NOT NULL,
      refund NUMERIC NOT NULL,

      CONSTRAINT pricing_pk PRIMARY KEY (id)
      );

      CREATE TABLE MEDICINE (
      id IDENTITY NOT NULL,
      ean TEXT NOT NULL,
      name TEXT NOT NULL,
      dose_id DECIMAL,
      form_id DECIMAL,
      pricing_id DECIMAL,
      package_id DECIMAL,
      ingredient_id DECIMAL,
      disease_id DECIMAL,

      FOREIGN KEY (dose_id) REFERENCES DOSE (id),
      FOREIGN KEY (form_id) REFERENCES FORM (id),
      FOREIGN KEY (pricing_id) REFERENCES PRICING (id),
      FOREIGN KEY (package_id) REFERENCES PACKAGE (id),
      FOREIGN KEY (ingredient_id) REFERENCES INGREDIENT (id),
      FOREIGN KEY (disease_id) REFERENCES DISEASE (id),

      CONSTRAINT medicine_pk PRIMARY KEY (id)
      );
    \end{lstlisting}
  \end{minipage}

\end{document}
