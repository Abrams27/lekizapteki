%& --translate-file latin2pl
\documentclass{article}

\usepackage{polski}
\usepackage{hyperref}
\usepackage[T1]{fontenc}
\usepackage[utf8]{inputenc}
\usepackage{listings}
\hypersetup{
colorlinks=true,
urlcolor=blue,
}
\setcounter{section}{-1}
\lstset{
basicstyle=\fontsize{6}{8}\selectfont\ttfamily
}


\title{
Lekizapteki\\
\large architektura}
\author{Marcin Abramowicz \and Mateusz Danowski \and Dawid Jamka \and Tomasz Patyna}


\begin{document}
  \maketitle

  \section{Dziennik zmian}
  \begin{tabular}{|c|c|c|}
    Nr iteracji & Data & Opis zmian \\
    \hline
    1. & 25.03.2020 & Utworzenie dokumentu oraz jego pierwsza wersja. \\
  \end{tabular}

  \section{Ogólna struktura}
  Aplikacja składa się z frontendu WWW napisanego w
  \href{https://angular.io}{Angularze}, backendu napisanego w
  \href{https://spring.io}{Spring Framewok} oraz bazy danych
  \href {https://spring.io}{H2}.

  \section{Komunikacja}
  \subsection{Frondend - Backend}
  Komunikacja odbywa się protokołem HTTP, serwer dostarcza
  \href{https://en.wikipedia.org/wiki/Representational_state_transfer}{RESTful API},
  obsługując rządania w formacie
  \href{https://en.wikipedia.org/wiki/JSON}{json}.
  Serwer jest dostepny na
  \href{http://students.mimuw.edu.pl:7312}{serwerze Students na porcie `7312`},
  natomiast frontend wykonuje zapytania za pomocą
  \href{https://angular.io/guide/http}{httpClient'a}.

  \subsubsection{Endpointy}

  \begin{lstlisting}
    GET /lekizapteki/diseases
    Accept: application/json

    Response:
    HTTP/1.1 200 (OK)
    Content-Type: application/json
    Body:
    [
      {
        "id": Long,
        "name":"String"
      }
    ]


    GET /lekizapteki/medicines
    Accept: application/json

    Parameters:
    diseaseId: Long (optional)

    Responses:
    HTTP/1.1 200 (OK)
    Content-Type: application/json
    Body:
    [
      {
        "ean": "String",
        "name": "String",
        "dose": "String"
      }
    ]

    HTTP/1.1 404 (Not Found)
    Content-Type: application/json
    Message: "No medicine on such disease"

    TODO?


    GET /lekizapteki/medicines/identical
    Accept: application/json

    Parameters:
    ean: String
    diseaseId: Long (optional)

    Responses:
    HTTP/1.1 200 (OK)
    Content-Type: application/json
    Body:
    [
      {
        "ean": "String",
        "name": "String",
        "dose": "String"
      }
    ]

    HTTP/1.1 404 (Not Found)
    Content-Type: application/json
    Message: "No medicine with such EAN"

    TODO

  \end{lstlisting}

  \subsection{Backend - Baza Danych}
  Backend komunikuje się z lokalną bazą danych dzięki
  \href{https://hibernate.org}{Hibernate} oraz
  \href{https://spring.io/projects/spring-data-jpa} {Spring Jpa}.

  \section{Frontend}

  \section{Backend}

  \section{Baza Danych}

  \begin{lstlisting}
    CREATE TABLE DISEASE (
    id                   IDENTITY            NOT NULL,
    name                 TEXT                NOT NULL,

    CONSTRAINT disease_pk PRIMARY KEY (id)
    );

    CREATE TABLE INGREDIENT (
    id                  IDENTITY            NOT NULL,
    name                TEXT                NOT NULL,

    CONSTRAINT ingredient_pky PRIMARY KEY (id)
    );

    CREATE TABLE DOSE (
    id                  IDENTITY            NOT NULL,
    dose                TEXT                NOT NULL,

    CONSTRAINT dose_pk PRIMARY KEY (id)
    );

    CREATE TABLE FORM (
    id                  IDENTITY            NOT NULL,
    name                TEXT                NOT NULL,

    CONSTRAINT form_pk PRIMARY KEY (id)
    );

    CREATE TABLE PACKAGE (
    id                  IDENTITY            NOT NULL,
    content             TEXT                NOT NULL,

    CONSTRAINT package_pk PRIMARY KEY (id)
    );

    CREATE TABLE PRICING (
    id                  IDENTITY            NOT NULL,
    salePrice           NUMERIC             NOT NULL,
    retailPrice         NUMERIC             NOT NULL,
    totalFunding        NUMERIC             NOT NULL,
    chargeFactor        NUMERIC             NOT NULL,
    refund              NUMERIC             NOT NULL,

    CONSTRAINT pricing_pk PRIMARY KEY (id)
    );

    CREATE TABLE MEDICINE (
    id                 IDENTITY            NOT NULL,
    ean                TEXT                NOT NULL,
    name               TEXT                NOT NULL,
    dose_id            DECIMAL,
    form_id            DECIMAL,
    pricing_id         DECIMAL,
    package_id         DECIMAL,
    ingredient_id      DECIMAL,
    disease_id         DECIMAL,

    FOREIGN KEY (dose_id) REFERENCES DOSE (id),
    FOREIGN KEY (form_id) REFERENCES FORM (id),
    FOREIGN KEY (pricing_id) REFERENCES PRICING (id),
    FOREIGN KEY (package_id) REFERENCES PACKAGE (id),
    FOREIGN KEY (ingredient_id) REFERENCES INGREDIENT (id),
    FOREIGN KEY (disease_id) REFERENCES DISEASE (id),

    CONSTRAINT medicine_pk PRIMARY KEY (id)
    );
  \end{lstlisting}

\end{document}
